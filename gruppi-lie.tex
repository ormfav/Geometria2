\begin{definition}[Gruppo di Lie]
    $G$ è gruppo di Lie se è varietà differenziabile dotata di due
    applicazioni lisce $\mu:G \times G \to G$, $\nu: G \to G$
    e un elemento $e$ tc $G$ sia gruppo con operazioni
    $g_1g_2:=\mu(g_1,g_2)$, $g^{-1}=\nu(g)$, $gg^{-1}=e$
\end{definition}
\begin{definition}[Omomorfismo tra gruppi di Lie]
    $\phi:G\to G$ tc sia omomorfismo tra gruppi e sia liscia.
\end{definition}
\begin{definition}[Sottogruppo di Lie]
    $H\subseteq G$ sottogruppo è sottogruppo di Lie se è anche
    sottovarietà.
\end{definition}

\begin{ex}(Gruppi su $\R$)\\
    $(\R,+)$ con $\mu(x,y)=x+y$, $\nu(x)=-x$.\\
    $(\R^{\times},\dot)$ con $\mu(x,y)=xy$, $\nu(x)=1/x$.\\
    $\exp:\R\to\R^{\times}$ è omomorfismo fra questi due gruppi.
\end{ex}
\begin{definition}[Traslazione a sinistra]
    $g\in G$ definisco traslazione a sinistra $L_g:G\to
    G:h\mapsto\mu(g,h)=gh$
\end{definition}
Proprietà:
\begin{itemize}
    \item Continuità
        \begin{prpr}
            $L_g$ è liscia
        \end{prpr}
        \begin{proof}
            Siccome $\mu$ liscia per definizione.
        \end{proof}
    \item Traslazione per l'elemento neutro
        \begin{prpr}
            $L_e=id_G$
        \end{prpr}
    \item Composizione
        \begin{prpr}
            $L_g\circ L_{g'}=L_{gg'}$
        \end{prpr}
        \begin{proof}
            $(L_g\circ L_{g'})(h)
            =L_g(L_{g'}(h))=g(g'h)=gg'(h)=L_{gg'}h$ dove ho usato
            l'associatività del gruppo.
        \end{proof}
    \item Inversa
        \begin{prpr}
            $L_g$ è diffeomorfismo
        \end{prpr}
        \begin{proof}
            Dalle precedenti proprietà segue che $L_g\circ
            L_{g^{-1}}=L_e=id_G=L_{g'}\circ L_g$ ovvero esiste
            l'inversa che è ancora una traslazione a sinistra e
            quindi è liscia
        \end{proof}
    \item Differenziale
        \begin{prpr}
            $\df{L_g}{h}$ è isomorfismo
        \end{prpr}
        \begin{proof}
            Dalle proprietà del differenziale siccome $L_g$ è
            diffeomorfismo.
        \end{proof}

\end{itemize}

\begin{definition}[Campi $X^v$]
    $v\in T_eG$ fissato $X^v$ campo vettoriale su $G$ è definito da
    $(X^v)g:=\df{L_g}{e}v\in T_gG$
\end{definition}
Osservazione: $X^v_e=(X^v)_e=\df{L_e}{e}v=id_{t_G}v$.

\begin{definition}[Campi invarianti a sinistra]
    $X$ campo vettoriale su $G$ è invariante a sinistra se
    $\df{L_g}{h}X_h=X_{L_g(h)}=X_{gh}\in T_{gh}G,\,\,\forall g,h\in
    G$.
\end{definition}
In altre parole significa che $X$ è $L_g$-correlato con sè stesso.
Esistono campi così?

\begin{theorem}
    $X$ è invariante a sinistra iff è nella forma $X^v$ per un
    certo $v\in T_eG$.
\end{theorem}
\begin{proof}
    Dimostro che ogni $X^v$ è invariante a sinistra ovvero che vale
    $\df{L_g}{h}X^v_h=X^v_{gh}$. Usando le proprietà del
    differenziale e della tralsazione a sinistra
    $\df{L_g}{h}X^v_h=\df{L_g}{h}(\df{L_h}{e}(v))=\df{L_g\circ
    L_h}{e}(v)=\df{L_{gh}}v=X^v_{gh}$.

    Dimostro che ogni campo invariante a sinistra (cioè
    $\df{L_g}{h}X_h=X_{gh}$) è del tipo $X^v$ per un certo $v\in
    T_eG$. Prendo nella definizione $h=e$, allora
    $\df{L_g}{e}X_e=X_{ge}=X_g$ ma $X_e\in T_eG$ e quindi
    $X_g=X^v_g$ con $v=X_e$.
    
\end{proof}

Da quanto detto abbiamo il seguente ragionamento.
$G$ gruppo di Lie su cui sono definiti $X,Y$ campi vettoriali
invarianti a sinistra, ovvero tc $X,Y$ sono $L_g$- correlati con
loro stessi $\forall g\in G$. Ma allora $\comm{X}{Y}$ è
$L_g$-correlato con sè stesso e quindi è invariante a sinistra.
\begin{itemize}
    \item $X$ invariante a sinistra $\iff$ $X=X^v$ per un certo
        $v\in T_eG$
    \item $Y$ invariante a sinistra $\iff$ $Y=X^w$ per un certo
        $w\in T_eG$
    \item $\comm{X}{Y}$ invariante a sinistra $\iff$
        $\comm{X}{Y}=X^z$ per un certo $z\in T_eG$
\end{itemize}
Le parentesi di Lie inducono un'operazione che associa a due
elementi di $T_eG$ un elemento di $T_eG$.

\begin{definition}[Prodotto di Lie o commutatore]
    $x,y,z\in T_eG$, il prodotto di Lie $\comm{v}{w}=z$ iff le
    parentesi di Lie $\comm{X^v}{X^w}=X^z$
\end{definition}

\begin{definition}[Algebra di Lie]
    Spazio vettoriale $\g$ dotato di applicazione bilineare
    $\g\times\g\to\g:(a,b)\mapsto \comm{a}{b}$ che goda delle
    proprietà 
    \begin{itemize}
        \item $\comm{a}{b}=-\comm{b}{a}$ 
        \item identità di Jacobi.
    \end{itemize}
\end{definition}

\subsection{Gruppi matriciali}
$GL(n,\R)=\left\{ A\in M_n(\R): detA\neq0 \right\}$ è un gruppo di
Lie con sottogruppi di Lie, ad esempio, $SO(n)=\left\{ A\in
M_n(\R): AA^t=id, detA=1 \right\}$ e 
$SL(n)=\left\{ A\in M_n(\R): detA=1 \right\}$.

Vediamo prima di tutto che $GL(n,\R)$ è un gruppo di Lie.
$GL(n,\R)\in M_n(\R)\cong\R^{n^2}$ e in particolare è un aperto:
essendo diffeomorfo ad un aperto di $\R^k$ è una varietà con
atlante $\left\{ id_{GL(n,\R)} \right\}$. Prendo come $\mu$ il
prodotto tra matrici, che è liscio in quanto polinomiale, e come
$\nu$ l'operazione che ad una matrice associa la sua inversa
rispetto al prodotto. $A^{-1}=\frac{1}{detA}A^{\#}$ con
$A^{\#}_{ij}=(-)^{i+j}$ per il determinante della sottomatrice
ottenuta cancellando da $A$ la $j$-esima riga e la $i$-esima
colonna. Siccome in $GL(n,\R)$ il determinante è sempre diverso da
zero questa operazione non da problemi ed è liscia.
