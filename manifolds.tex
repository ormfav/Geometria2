\begin{definition}[Carta]
    $M$ insieme. Una carta di $M$ è la coppia $(U,x)$ con:
    \begin{itemize}
        \item $U\subseteq M$ 
        \item $x: U \rightarrow x(U)\subseteq \R^m$
        biiezione\footnote{$\exists x^{-1}: x(U)\rightarrow U$}
        e $x(U)$ aperto\footnote{$\forall u\in U\exists
        \epsilon\geq 0$ tc $B_{\epsilon}(u)=\{x\in \R^m:
        \norm{x-u}\leq\varepsilon$\}. Segue che $U=\bigcup_{u\in
        U} B_{\epsilon}(u)$}. $x$ è chiamata parametrizzazione di
        $U$.
    \end{itemize}
\end{definition}
\begin{definition}[Carte compatibili]
    Le carte $(U_{\alpha},x_{\alpha}),(U_{\beta},x_{\beta})$ sono
    compatibili se $F_{\beta\alpha}=x_{\beta}\circ
    x^{-1}_{\alpha}:x_{\alpha}(U_{\alpha}\cap U_{\beta})\rightarrow
    \R^m$ è diffeomorfismo\footnote{Mappa liscia con inversa
    liscia}.
\end{definition}
TODO:DISEGNO 220301
Osservazioni: 
\begin{itemize}
    \item Una carta è sempre compatibile con sè stessa 
    \item l'inversa di $F$ esiste sempre ma non è detto sia liscia
\end{itemize}
\begin{definition}[Atlante]
    $\A=\left\{ (U_{\alpha},x_{\alpha}) \right\}_{\alpha\in I}$
    collezione di carte di M a due a due compatibili tc
    $M=\bigcup_{\alpha\in I}U_{\alpha}$
\end{definition}
\begin{definition}[Varietà differenziabile]
    Coppia $(M,\A)$ con $x_{\alpha}(U_{\alpha})\subseteq \R^m$
    aperto $\forall\alpha\in I$.
\end{definition}

\begin{ex}
    $M=\R^m$.\\
    L'unica carta è $x=id_{\R^m}:U=M=\R^m\rightarrow\R^m$.\\ 
    $id(M)$ aperto essendo immagine di aperto.\\
    $\A=\left\{ (M,id_M) \right\}$.\\
    Quindi $\R^m$ è varietà.
\end{ex}
\begin{ex}
    $M\subset\R^m$ aperto.\\
    L'unica carta è $x=id_{M}:U=M\rightarrow\R^m$.\\ 
    $id(M)$ aperto essendo immagine di aperto.\\
    $\A=\left\{ (M,id_M) \right\}$.\\
    Quindi ogni aperto di $\R^m$ è varietà.
\end{ex}
\begin{ex}
    $M=S^n$ (sfera n-dimensionale).
    TODO: DA FARE 220301
\end{ex}



Dato che esiste una definizione di mappe lisce su $\R^m$ posso
introdurre grazie all'atlante una definizione di mappe lisce su
varietà.
\begin{definition}[Applicazione liscia]
    $M,N$ varietà differenziabili di dimensione $m,n$.
    $f:M\rightarrow N$ è liscia in $p\in M$ se $\forall (U,x), (V,y)$
    carte di $M,N$ con $p\in U,f(p)\in V$ la funzione  $F=y\circ
    f\circ x^{-1}: x(U\cap f^{-1}(V))\rightarrow\R^n$ è\footnote{In
    generale $f(U)\neq V$, per questo si prende l'intersezione}
    liscia in $x(p)$ con $x(U\cap f^{-1}(V))$ aperto di $\R^m$.
\end{definition}
TODO:DISEGNO 220308
Verificare la proprietà per ogni carta è problematico
\begin{obs}
    È suffuciente effettuare la verifica per una sola coppia di
    carte.
\end{obs}
\begin{proof}
    Supponiamo $F=y\circ f\circ x^{-1}$ liscia, bisogna mostrare
    che $w\circ f\circ v^{-1}$ sia liscia.
    $w\circ f\circ v^{-1}=w\circ (y^{-1}\circ y)\circ f\circ
    (x^{-1}\circ x)\circ v^{-1}=(w\circ y^{-1})\circ (y\circ f\circ
    x^{-1})\circ (x\circ v^{-1})$ che è composizione di F (liscia
    per ipotesi) e di due cambi di carta che sono diffeomorfisimi e
    quindi lisci.
    TODO:DISEGNO 220301
\end{proof}
\begin{obs}
    $M,N,S$ varietà, $f:M\rightarrow N, g:N\rightarrow S$ lisce.
    $g\circ f:M\rightarrow S$ liscia.
\end{obs}
\begin{proof}
    Sia $p\in M$, carte $(U,x),(V,y),(W,z)$ di $M,N,S$ tc $p\in
    U,f(p)\in V,g(f(p))\in W$. Devo mostrare che $z\circ(g\circ
    f)\circ x^{-1}$ è liscia su $(g\circ f)(U\cap W)$.
    TODO:DISEGNO
    $z\circ(g\circ f)\circ x^{-1}=z\circ g \circ (y^{-1}\circ y)
    \circ f\circ x^{-1}=(z\circ g \circ y^{-1})\circ (y \circ
    f\circ x^{-1})$ composizione di funzioni lisce per l'ipotesi
    $f,g$ lisce.
\end{proof}

\begin{ex}
    $M$ varietà, $N=\R^n$ (l'unica carta è $(\R^n,id_{\R^n})$).\\
    $f:M\rightarrow \R^n$ è liscia se $id_{\R^n}\circ f\circ
    x^{-1}=f\circ x^{-1}$ è liscia.
\end{ex}
\begin{ex}
    Una parametrizzazione è liscia? $(U,x)$ carta di $M$.
    $x:U\rightarrow \R^m$ per quanto visto è una mappa fra le
    varietà $U$ con altante $\left\{ (U,x) \right\}$ e $\R^m$.
    Dalla quanto visto prima $x$ è liscia se $x \circ x^{-1}$ è
    liscia. Ma si tratta dell'identità e quindi di una funzione
    liscia.
    
\end{ex}
\subsection{Fibre}
La sfera è un insieme del tipo $F^{-1}(b)$ con $F=\sum x_i^2$ e
$b=1$. Esiste un modo semplice per capire se insiemi di questo tipo
sono varietà, la cui dimostrazione sfrutta il teorema della
funzione inversa\footnote{TODO: TEOREMA FUNZIONE INVERSA}.
\begin{theorem}
    $U\subseteq\R^{m+n}$ aperto\\
    $F:U\rightarrow \R^m$ liscia\\
    !!$b\in\R^m$ tc $rkJ_aF$ massimale per ogni $a=F^{-1}(b)$\\
    allora:\\
    $F^{-1}(b)=\left\{ x\in U: F(x)=b \right\}$ è varietà di
    dimensione n\\
TODO: CARTE
\end{theorem}
\begin{proof}
    TODO: PROOF 220301
\end{proof}

\begin{obs}
    $F:\R^{n+m}\rightarrow\R^m$ sommersione su $F^{-1}(b)$,
    $f:\R^{n+m}\rightarrow \R$ lisca. Allora
    $f\eval_{M}:M=F^{-1}(b)\rightarrow\R$ è liscia. Infatti deve
    essere liscia $f\circ x^{-1}$ ma $x$ abbiamo visto essere
    liscia. ???è vero???
\end{obs}
