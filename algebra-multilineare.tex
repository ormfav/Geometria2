\begin{definition}[k-forme alternanti]
    $V$ sv su $\R$. Una k-forma alternante è
    un'applicazione $f:V\times V\times ...\times V\equiv V^k\to\R$
    lineare in ogni variabile e tc $f(v_1,\dots,v_k)=0$ se
    $v_i=v_j$ per $i\neq j$.
\end{definition}
\begin{obs}
    $f(v_1,\dots,v_k)=0$ se $v_i=v_j$ per $i\neq j$ equivale a
    $f(v_1,\dots,v_k)=
    \epsilon(\sigma)f(v_{\sigma_1},\dots,v_{\sigma_k})$.
\end{obs}
\begin{proof}
    Verifico doppia implicazione.

    ($\Rightarrow$) Suppongo $v_i=v_j=v+w$ per certe $i\neq j$.
    Per la multilinearità $0=f(\dots v+w\dots v+w\dots)=
    f(\dots v\dots v\dots)+f(\dots v\dots w\dots)+
    f(\dots w\dots v\dots)+f(\dots w\dots w\dots)=
    f(\dots v\dots w\dots)+f(\dots w\dots v\dots)$ ovvero
    $f(\dots v\dots w\dots)=-f(\dots w\dots v\dots)$: lo scambio
    porta un $-$. $S_k$ gruppo permutazioni è generato dagli scambi
    e $\epsilon:S_k\to\left\{ \pm 1 \right\}$ è omomorfismo, quindi
    la tesi.

    ($\Leftarrow$) $f(\dots v\dots v\dots)=-f(\dots v\dots v\dots)$
    dato che lo scambio lascia inalterata $f$, ma quindi se due
    elementi sono uguali $f=0$.
\end{proof}

\begin{ex}
    $V=\R^n$, $det:\R^{n^2}\to\R:(v_1\dots v_n)\mapsto det(v_1,\dots
    v_n)=det(v_1|v_2\dots| v_n)$ è una n-forma alternante.
\end{ex}

\begin{definition}
    $Alt^k(V)$ insieme delle k-forme alternanti su $V$
\end{definition}
Con le operazioni $(\lambda f + \mu g)(v_1,\dots v_k)=\lambda
f(v_1\dots v_k)+ \mu g(v_1\dots v_k)$, $Alt^k(V)$ è sv.
Ha quindi senso studiarne la dimensione e cercarne una base. Alcune
osservazioni.
\begin{itemize}
    \item $Alt^0(V)=\R$ per definizione 
    \item $Alt^1(V)=V^*$ (duale: tutte le mappe lineari $V\to\R$)
    \item $Alt^k(V)=\left\{ 0 \right\}$ se $k>dimV=n$, infatti:
        siccome $k>n$ $v_1\dots v_k$ sono necessariamente
        linearmente dipendenti. Supponiamo $v_k=\sum_{i=1}^{k-1}c_i
        v_i$, allora per ogni k-forma si ha $f(v_1\dots
        v_k)=\sum_{i=1}^{k-1}c_i f(\dots v_i\dots v_i)=0$ 
\end{itemize}

\begin{lem}
    $dim(Alt^k(V))\leq\binom{n}{k}$ se $k\leq n$.
\end{lem}
\begin{proof}
    scelta $e_1,\dots e_n$ base di $V$, $v_1,\dots v_k\in V$ si
    scrivono $v_j=\sum_{i_j=1}^n v_{ji_j}e_{i_j}$ (il pedice
    alle $i$ serve solo a ricordare a quale vettore sto facendo
    riferimento ed è utile nel prossimo passaggio). 
    $f(v_1,\dots v_k)=
    \sum_{i_1}^n v_{1,i_1}f(e_{i_1},v_2\dots v_k)=
    \sum_{i_1,i_2}^n v_{1,i_1}v_{2,i_2}f(e_{i_1},e_{i_2}, 
    v_3\dots v_k)=\dots =
    \sum_{i_1\dots,i_k}^n v_{1,i_1}\dots v_{k,i_k} 
    f(e_{i_1},\dots e_{i_k})$. Posso considerare solo i termini
    della sommatoria in cui tutti gli $e_{i_j}$ sono diversi
    (negli altri casi $f$ si annulla) e posso quindi
    riordinarli con una permutazione $\sigma$ tc $i\leq
    i_{\sigma_1}<i_{\sigma_2}<\dots< i_{\sigma_k}\leq n$.
    Perciò $f$ è determinata da al massimo $\binom{n}{k}$ costanti.
\end{proof}
Se trovo un insieme di $\binom{n}{k}$ k-forme linearmente
indipendenti ho base e dimensione. Per farlo introduco
l'operazione: 
\begin{definition}[Prodotto esterno o wedge]
    $\wedge: Alt^r(V)\times Alt^s(V)\to Alt^{r+s}(V):(f,g)\mapsto
    f\wedge g$ tc
    \begin{itemize}
        \item $f\wedge g=(-)^{rs}g\wedge f$
        \item $f\wedge(g\wedge h)=(f\wedge g)\wedge h$.
    \end{itemize}
\end{definition}
\begin{obs}
    $f\in Alt^1(V)$ allora $f\wedge f=0$
\end{obs}
\begin{obs}
    $f_1,\dots f_k \in Alt^1(V)$ allora 
    $f_1\wedge\dots\wedge f_k =
    \epsilon(\sigma)f_{\sigma_1}\wedge\dots\wedge f_{\sigma_k}$
\end{obs}
\begin{obs}
    $f_1,\dots f_k \in Alt^1(V)=V^*$ allora il prodotto esterno è
    una k-forma e in particolare
    $(f_1\wedge\dots\wedge f_k)(v_1,... v_k)=det(f_j(v_i))$ 
\end{obs}

Prendo $I=\left\{ i_1,\dots i_k \right\}\subseteq 
\left\{ 1,\dots n \right\}$ ordinato e $v_1\dots v_k\in V$.
Prendo poi $e_1\dots e_n$ e $\epsilon_1\dots\epsilon_n$ basi
di $V,V^*$ (cioè $\epsilon_i(e_j)=\delta_{ij}$) e definisco
$\epsilon_I=e_{i_1}\wedge\dots\wedge e_{i_k}$. Ho che 
$v_j=\sum_{i_j=1}^n v_{j i_j}e_{i_j}$ e 
$\epsilon_{i_j}(v_j)=v_{j i_j}$. Ne segue che, definite
$\epsilon_I=e_{i_1}\wedge\dots\wedge e_{i_k}$ al variare di $I$,
queste hanno la forma
\begin{displaymath}
    \epsilon_I(v_1,\dots v_k)= 
    \det\mqty
    (\epsilon_{i_1}(v_1)  & \dots & \epsilon_{i_1}(v_k) \\
    \vdots                &       & \vdots              \\
    \epsilon_{i_k}(v_1)  & \dots & \epsilon_{i_k}(v_k)  )=
    \det\mqty
    (v_{1i_1}  & \dots & v_{ki_1} \\
    \vdots     &       & \vdots   \\
    v_{1i_k}   & \dots & v_{ki_k} )
\end{displaymath}
Ho $\binom{n}{k}$ possibili scelte di $I$: ho costruito una
famiglia di $\binom{n}{k}$ k-forme. DEVO MOSTRARE SONO LI



Si scrive $Alt^k(V)=\bigwedge^kV^*$

