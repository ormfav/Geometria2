Definizioni di derivata su varietà. 
$p\in M$ voglio fare somma e prodotto di $f:U\rightarrow\R$,
$g:V\rightarrow\R$ con $p\in U\cap V$ e $f,g$ lisce. Problema: 
se $V\neq U$ somma e prodotto sono definite solo su $U\cap V$.
Voglio togliere riferimenti al dominio. 
Definisco l'algebra dei germi $C^\infty(M,p):=\left\{ (U,f): p\in
U, f:U\rightarrow\R\text{ liscia} \right\}/_\sim$ con la relazione
di equivalenza $(U,f)\sim(V,g)$ se $f=g$ su $W=U\cap V,\,\,p\in W$.
La classe di equivalenza di $(U,f)$ è detta germe di $f$. In pratica
mi preoccupo solo del comportamento delle funzioni molto vicino a
$p$. Posso quindi scrivere
$[(U,f)]_\sim+[(V,g)]_\sim=[(U,f)+(V,g)]_\sim$ e
$[(U,f)]_\sim[(V,g)]_\sim=[(U,f)(V,g)]_\sim$. Per semplicita da qui in avanti
$[(U,f)]_\sim\equiv f$. 

Posso introdurre lo spazio tangente in due modi equivalenti:
tramite derivazioni e tramite cammini.

\subsection{Derivazioni}
\begin{definition}[Derivazione]
    Applicazione $v:C^\infty(M,p)\rightarrow\R$ tc\footnote{Vanno
    lette: $v(\lambda [(U,f)]_\sim+\mu [(V,g)]_\sim)=\lambda
    v([(U,f)]_\sim)+\mu v([(V,g)]_\sim)$ e
    $v([(U,f)]_\sim[(V,g)]_\sim)=
    v([(U,f)]_\sim)[(V,g)]_\sim+[(U,f)]_\sim v([(V,g)]_\sim)$}
    \begin{itemize}
        \item lineare: $v(\lambda f+\mu g)=\lambda v(f)+\mu v(g)$
        con $\lambda,\mu\in\R$ 
        \item soddisfa leibniz: $v(fg)=v(f)g+fv(g)$
    \end{itemize} 
\end{definition}
Queste mappe esistono?
Un esempio immediato: 
\begin{ex}
    Carta $(U,x)$ di $M$, $x:x(U)\rightarrow\R^m$. Se $f$ è liscia
    $x(p)$ è liscia in intorno di $x(p)\in x(U)$ per definizione.
    Posso quindi avere una derivazione sui germi di questa funzione
    composta. Definisco
    \begin{equation*}
        \Biggl(\eval{\pdv{x_i}}_p\Biggr)(f):=\Biggl(\pdv{f\circ
        x^{-1}}{t_i}\Biggr)(x(p))
    \end{equation*}
    con $t_i$ le coord su $\R^m$. Si verifica facilmente che sono
    derivazioni.
\end{ex}
L'insieme delle derivazioni su $C^\infty(M,p)$ è spazio vettoriale
su campo $\R$ con somma tra vettori e prodotto per scalare definiti
in modo naturale.
\begin{obs}
    $\{\pdv{x_i}|_p\}_{i=1}^m$ base di $T_pM$ e
    quindi dim$T_pM=m=$dim$M$\footnote{Il concetto di
    dimensione per spazi vettoriali e per varietà è definito in
    modo diverso.}
\end{obs}
\begin{theorem}
    $v\in T_pM$, $(x,U)$ carta di $M$ $v=\sum_i^m a_i\pdv{x_i}|_p$
    con
    \begin{displaymath}
        a_i=v(x_i)
    \end{displaymath}
\end{theorem}
\begin{proof}
    Osservo: 
    \begin{displaymath}
        \Biggl(\eval{\pdv{x_i}}_p\Biggr)(x_j)=\Biggl(\pdv{x_j\circ
        x^{-1}}{t_i}\Biggr)(x(p))
    \end{displaymath}
    e $((x_1\dots x_m)\circ x^{-1})=(x\circ x^{-1})(t)=t=(t_1\dots
    t_m)$. Quindi
    \begin{displaymath}
        \Biggl(\eval{\pdv{x_i}}_p\Biggr)(x_j)=
        \Biggl(\pdv{t_j}{t_i}\Biggr)(x(p))=\delta_{ij}
    \end{displaymath}
    $v(x_j)=\sum_i^m a_i\pdv{x_i}|_p(x_j)=a_j$
\end{proof}

\begin{theorem}
    $v\in T_pM$, $(U,x),(V,y)$ carte di $M$, $v=\sum_i^m
    a_i\pdv{x_i}|_p=\sum_i^m b_i\pdv{y_i}|_p$ con 
    \begin{displaymath}
        a=J_{y(p)}F\cdot b
    \end{displaymath}
    dove $F=x\circ y^{-1}$ è il cambio di coordinate.
\end{theorem}
\begin{proof}
    Per quanto visto nella dimostrazione prima e usando la
    definzione degli elementi della base:
    \begin{displaymath}
        a_j=\sum b_i\Biggl(\eval{\pdv{y_i}}_p\Biggr)(x_j)=
        \sum b_i \Biggl(\pdv{x_j\circ y^{-1}}{t_i}\Biggr)(y(p))
    \end{displaymath}
    che chiamando $F=x\circ y^{-1}$ è proprio il prodotto
    matrice-vettore.
\end{proof}

Vgliamo legare spazi tangenti di $M,N$ collegate da mappa liscia
$\Phi$.\\
Problema: devo far agire sui germi di $N$ derivazioni che agiscono
sui germi di $M$. Devo prima di tutto introdurre un oggetto che mi
faccia passare da
$C^\infty(N,\Phi(p))$ a
$C^\infty(M,p)$.
\begin{definition}[Pull-back]
    \begin{displaymath}
        \Phi^*:C^\infty(N,\Phi(p))\rightarrow C^\infty(M,p):
        g\mapsto \Phi^*(g):=g\circ\Phi
    \end{displaymath}
\end{definition}
\begin{obs}
    $\Phi^*$ è lineare: $\Phi^*(\lambda g+\mu h)=\lambda\Phi^*(g)+
    \mu\Phi^*(h)$
\end{obs}
\begin{proof}
    Devo mostrare che l'identità vale $\forall x\in M$.  
    $\Phi^*(\lambda g+\mu h)(x)=(\lambda g+\mu h)(\Phi(x))= 
    \lambda g(\Phi(x))+\mu h(\Phi(x))=\lambda \Phi^*(g)(x)+\mu
    \Phi^*(h)(x)$
\end{proof}
\begin{definition}[Differenziale]
    $\df{\Phi}{p}:T_pM\rightarrow T_{\Phi(p)}N:v\mapsto
    \df{\Phi}{p}(v):=v(\Phi^*(g))(=v(g\circ\Phi))$
\end{definition}
Proprietà. 
\begin{itemize}
    \item È mappa fra spazi vettoriali: vogliamo sia lineare
        \begin{prpr}
            $\df{\Phi}{p}(\lambda v +\mu
            u)=\df{\Phi}{p}(v)+\df{\Phi}{p}(u)$
        \end{prpr}
        \begin{proof}
            L'ugugaglianza vale iff $(\df{\Phi}{p}(\lambda v +\mu
            u))(g)=(\df{\Phi}{p}(v))(g)+(\df{\Phi}{p}(u))(g)\quad\forall
            g\in C^\infty(N,\Phi(p))$. Dalla definizione di spazio tangente 
            $(\df{\Phi}{p}(\lambda v +\mu u))(g)=(\lambda v +\mu
            u)(g\circ\Phi)=\lambda v(g\circ\Phi)+\mu u(g\circ u)=
            (\df{\Phi}{p}(v))(g)+(\df{\Phi}{p}(u))(g)$.
        \end{proof}
        
    \item differenziale dell'identità
        \begin{prpr}
            $\df{id_M}{p}=id_{T_pM}$
        \end{prpr}
        \begin{proof}
            $v(g\circ id_M)=v(g)\quad\forall g$
        \end{proof}

    \item Composizione di differenziali
        \begin{prpr}
            $\Phi:M\rightarrow N,\,\Psi:N\rightarrow S$ lisce,
            $p\in M$ allora $\df{\Psi}{\Phi(p)}\circ\df{\Phi}{p}=
            \df{(\Psi\circ\Phi)}{p}:T_pM\rightarrow
            T_{\Psi(\Phi(p))}S$
        \end{prpr}
        \begin{proof}
            Vale iff $\df{(\Psi\circ\Phi)}{p}(v)=
            (\df{\Psi}{\Phi(p)}\circ\df{\Phi}{p})(v)\quad\forall
            v\in T_pM$. Fissata $v$ vale iff
            $(\df{(\Psi\circ\Phi)}{p}(v))(h)=
            ((\df{\Psi}{\Phi(p)}\circ\df{\Phi}{p})(v))(h)\quad
            \forall h\in C^\infty(S,\Psi(\Phi(p)))$. 
            $(\df{(\Psi\circ\Phi)}{p}(v))=v(h\circ(\Psi\circ\Phi))=
            v((h\circ\Psi)\circ\Phi)=(\df{\Phi}{p}(v))(h\circ\psi)= 
            ((\df{\Psi}{\Phi(p)}(\df{\Psi}{p})(v))(h)$.
        \end{proof}

    \item Differenziale dell'inversa
        \begin{prpr}
            Se $\Phi:M\rightarrow N$ diffeo con inversa $\Psi$
            allora $\df{\Psi}\Phi{p}:T_pM\rightarrow T_{\Phi(p)}N$
            è isomorfismo con inversa $\df{\Psi}{\Phi(p)}$. 
        \end{prpr}
        \begin{proof}
            $\Psi\circ\Phi=id_M$. Per la p2
            $\df{(\Psi\circ\Phi)}{p}=id_{T_pM}$. Per la p3
            $\df{\Psi}{\Phi(p)}\circ\df{\Phi}{p}=id_{T_pM}$.
            Analogamente per la composizione inversa.
        \end{proof}

\end{itemize}   

Siccome il differenziale è lineare vogliamo ottenerne la forma
matriciale
\begin{theorem}
    $(U,x),(V,y)$ carte di $M,N$. $\Phi_M\rightarrow N$ liscia.
    Su questa base
    \begin{displaymath}
        \df{\Phi}{p}=J_{x(p)}F
    \end{displaymath}
    con $F=y\circ\Phi\circ x^{-1}$
\end{theorem}
\begin{proof}
    Il differenziale applicato a un elemento della base di $T_pM$ è
    un elemento della base di $T_{\Phi(p)}N$, quindi
    $\df{\Phi}{p}(\pdv{x_j})|_p=\sum^nv_{ij}\pdv{y_i}|_{\Phi(p)}$
    con $v_{ij}\in\R$.
    Per quanto visto $v_{ij}=\df{\Phi}{p}(\pdv{x_j})|_p(y_i)=
    (\pdv{x_j})(y_i\circ\Phi)=\pdv{y_i\circ\Phi\circ
    x^{-1}}{t_j}(x(p))$. Ponendo $F=(F_1\dots F_n)$ con
    $F_i=y_i\circ\Phi\circ x^{-1}$ la tesi.
\end{proof}


\subsection{Cammini}
$\gamma:(-\varepsilon,\varepsilon)\rightarrow M$, $\gamma(0)=p$
cammino.
\begin{definition}[Vettore tangente]
    $\gamma_*(f):=\dv{f\circ\gamma}{\tau}\,(0)$ con
    $\tau\in(-\varepsilon,\varepsilon)$.
\end{definition}
$f\circ\gamma:\R\to\R$: ho una derivata nel senso comune del
termine.

\begin{definition}[Differenziale]
    $\gamma$ cammino, $\Phi:M\rightarrow N$ liscia.
    \begin{displaymath}
        \df{\Phi}{\gamma_*}:=(\Phi\circ\gamma)_*
    \end{displaymath}
    cioè il vettore tangente ottenuto mappando il cammino su $N$
    mediante $\Phi$.
\end{definition}

Ogni $v\in T_pM$ è del tipo $\gamma_*$ per un certo $\gamma$.
\begin{ex}
    $T_p\R^m=\R^m$ infatti:\\
    Carta $(\R^m,id_{\R^m})$. $v=\sum_i^m
    a_i\pdv{t_i}|_p\leftrightarrow a=(a_1\dots a_m)$ \\
    Preso $\gamma:(-\varepsilon,\epsilon)\rightarrow\R^m$,
    con $p=\gamma(0)$
    $\gamma_*(f)=
    (\dv{f(\gamma_1(\tau)\dots\gamma_m(\tau))}{\tau})|_{\tau=0}=
    \sum \pdv{f}{t_i}(p)\gamma_i'(0)$ e quindi per confronto
    $\gamma_i'(0)=a_i$.
\end{ex}

La definizione di differenziale ottenuta a partire dalle
derivazioni deve essere la stessa di quella ottenuta a partire dai
cammini. 
\begin{obs}
    Le due definizioni di differenziale sono equivalenti
\end{obs}
\begin{proof}
    Per la definizione data usando le derivazioni
    $(\df{\gamma_*}{p})(g)=\gamma_*(g\circ\Phi)=
    \dv{g\circ\Phi\gamma}{\tau}|_{\tau=0}=
    \dv{g\circ(\Phi\gamma)}{\tau}|_{\tau=0}=(\Phi\circ\gamma)_*(g)$
    e questo vale per tutte le $g$.
\end{proof}

\subsection{Fibre}
\begin{theorem}
    $M=F^{-1}(b)\subseteq\R^{n+m}$ con $F:\R^{n+m}\to\R^n$ allora
    $\forall a\in M$ $T_aM\equiv ker(\df{F}{a})\cong ker(J_aF)$
    dove $\df{F}{a}:T_a\R^{n+m}\to T_b\R^n$ e
    $J_aF:\R^{n+m}\to\R^n$.
\end{theorem}
\begin{proof}
    $rk(J_aF)$ è massimo e quindi uguale a $n$. Per il teorema di
    nullità+rango $dim(ker(F_aF))=n+m-n=m=dim(T_aM)$. Resta da
    mostrare che $T_aM\subseteq ker(J_aF)\cong ker(\df{F}{a})$.
    Sia $v\in T_aM$, prendo $\gamma:(-\epsilon,\epsilon)\to
    M:0\mapsto a$ tc $\gamma_*=v$
    ($\gamma_*(f)=\pdv{f\circ\gamma}{\tau}|_{\tau=0}$) allora 
    $(\df{F}{a}\gamma_*)(g)=\gamma_*(g\circ F)$ con $g\in
    C^\infty(\R^n,b)$. Su $M$ $F(a)=b$ costante e quindi
    $(\df{F}{a}v)(g)=(\df{F}{a}\gamma_*)(g)=\gamma_*(g\circ
    F)=\gamma_*(g(b))=0$ ovvero $v\in ker(\df{F}{a})$

    
\end{proof}

\begin{ex}[Spazio tangente a $S^n$]
    $S^n=F^{-1}(1)$ con $F:\R^{n+1}\to\R:(x_0\dots
    x_n)\mapsto\langle x,x\rangle$.
    $T_xS^n=ker(J_xF)=\left\{ y\in\R^{n+1}:\langle x,y\rangle=0
    \right\}$ (dato che $J_xF=2(x_0,\dots x_n)$ ovvero il fatto
    noto che i tangenti alla sfera sono ortogonali al raggio.
\end{ex}


